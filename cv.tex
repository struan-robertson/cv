%-------------------------
% Rezume, a latex resume template for developers
% https://www.overleaf.com/latex/templates/rezume/kfrvqywfkwjs
% Author : Nanu Panchamurthy
% Based off of: https://github.com/sb2nov/resume
% License : MIT

% Hope this resume template helps you land an awesome job. If you found this helpful, please consider starring the github repo here, .
%-------------------------



%------------PACKAGES----------------
\documentclass[a4paper,12pt]{article}

\usepackage{verbatim} % re-implements the "verbatim" and "verbatim*" environments

\usepackage{titlesec} % provides an interface to sectioning commands i.e. custom elements

\usepackage{color} % provides both foreground and background color management

\usepackage{enumitem} % provides control over enumerate, itemize and description

\usepackage{fancyhdr} % provides extensive facilities for constructing headers, footers and also controlling their use

\usepackage{tabularx} % defines an environment tabularx, extension of "tabular" with an extra designator x, paragraph like column whose width automatically expands to fill the width of the environment

\usepackage{latexsym} % provides mathematical symbols

\usepackage{marvosym} % provides martin vogel's symbol font which contains various symbols

\usepackage[empty]{fullpage} % sets margins to one inch and removes headers, footers etc..

\usepackage[hidelinks]{hyperref} % removes color and shadow of hyperlinks

\usepackage[normalem]{ulem} % provides "\ul" (uline) command which will break at line breaks

\usepackage[english]{babel} % provides culturally determined typographical rules for wide range of languages
%-----------------------------------------

\input glyphtounicode % converts glyph names to unicode
\pdfgentounicode=1 % ensures pdfs generated are ats readable

%----------FONT OPTIONS-------------------
\usepackage[default]{sourcesanspro} % uses the font source sans pro
\urlstyle{same} % changes url font from default urlfont to font being used by the document
%-----------------------------------------


%----------MARGIN OPTIONS-----------------
\pagestyle{fancy} % set page style to one configured by fancyhdr
\fancyhf{} % clear all header and footer fields

\renewcommand{\headrulewidth}{0in} % sets thickness of linerule under header to zero
\renewcommand{\footrulewidth}{0in} % sets thickness of linerule over footer to zero

\setlength{\tabcolsep}{0in} % sets thickness of column separator in tables to zero

% origin of the document is one inch from the top and from and the left
% oddsidemargin and evensidemargin both refer to the left margin
% right margin is indirectly set using oddsidemargin
\addtolength{\oddsidemargin}{-0.5in}
\addtolength{\topmargin}{-0.5in}

\addtolength{\textwidth}{1.0in} % sets width of text area in the page to one inch
\addtolength{\textheight}{1.0in} % sets height of text area in the page to one inch

\raggedbottom{} % makes all pages the height of current page, no extra vertical space added
\raggedright{} % makes all pages the width of current page, no extra horizontal space added
%------------------------------------------


%--------SECTIONING COMMANDS---------
% \titleformat{&lt;command&gt;}
%   [&lt;shape&gt;]{&lt;format&gt;}{&lt;label&gt;}{&lt;sep&gt;}
%     {&lt;before-code&gt;}[&lt;after-code&gt;]

% command is the sectioning command to be redefined
% shape is the style of the font; scshape stands for small caps style
% format is the format to be applied to whole title- label and text; absent here
% label defines the label
% sep is the horizontal separation between label and title body
% before-code is the code to be executed before
% after-code is the code to be executed after

\titleformat{\section}
  {\scshape\large}{}
    {0em}{\color{blue}}[\color{black}\titlerule\vspace{0pt}]
%-------------------------------------


%--------REDEFINITIONS----------------
% redefines the style of the bullet point
\renewcommand\labelitemii{$\vcenter{\hbox{\tiny$\bullet$}}$}

% redefines the underline depth to 2pt
\renewcommand{\ULdepth}{2pt}
%-------------------------------------


%--------CUSTOM COMMANDS--------------
%\vspace{} defines a vertical space of given size, modifying this in custom commands can help stretch or shrink resume to remove or add content

% resumeItem renders a bullet point
\newcommand{\resumeItem}[1]{
  \item\small{#1}
}

% commands to start and end itemization of resumeItem, rightmargin set to 0.11in to avoid the overflow of resumetItem beyond whatever resumeItemHeading is being used
\newcommand{\resumeItemListStart}{\begin{itemize}[rightmargin=0.11in]}
\newcommand{\resumeItemListEnd}{\end{itemize}}

% resumeSectionType renders a bolded type to be used under a section, used as skill type here, middle element is used to keep ":"s in the same vertical line
\newcommand{\resumeSectionType}[2]{
  \item\begin{tabular*}{0.96\textwidth}[t]{
    p{0.21\linewidth}p{0.79\linewidth}
  }
    \textbf{#1} & #2
  \end{tabular*}\vspace{-2pt}
}

% resumeTrioHeading renders three elements in three columns with second element being italicized and first element bolded, can be used for projects with three elements
\newcommand{\resumeTrioHeading}[3]{
  \item\small{
    \begin{tabular*}{0.96\textwidth}[t]{
      l@{\extracolsep{\fill}}c@{\extracolsep{\fill}}r
    }
      \textbf{#1} & \textit{#2} & #3
    \end{tabular*}
  }
}

% resumeTrioHeading renders three elements in three columns with second element being italicized and first element bolded, can be used for projects with three elements
\newcommand{\resumeRefereeHeading}[4]{
 \item\begin{tabular*}{0.96\textwidth}[t]{
    p{0.19\linewidth}p{0.81\linewidth}
  }
         \textbf{#1} & #2 \\
         \textit{\small #3} & #4 \\
  \end{tabular*}\vspace{-2pt}
}

% resumeQuadHeading renders four elements in a two columns with the second row being italicized and first element of first row bolded, can be used for experience and projects with four elements
\newcommand{\resumeQuadHeading}[4]{
  \item
  \begin{tabular*}{0.96\textwidth}[t]{l@{\extracolsep{\fill}}r}
    \textbf{#1} & #2 \\
    \textit{\small#3} & \textit{\small #4} \\
  \end{tabular*}
}

% resumeQuadHeadingChild renders the second row of resumeQuadHeading, can be used for experience if different roles in the same company need to added
\newcommand{\resumeQuadHeadingChild}[2]{
  \item
  \begin{tabular*}{0.96\textwidth}[t]{l@{\extracolsep{\fill}}r}
    \textbf{\small#1} & {\small#2} \\
  \end{tabular*}
}

% commands to start and end itemization of resumeQuadHeading, lefmargin for left indent of 0.15in for resumeItems
\newcommand{\resumeHeadingListStart}{
  \begin{itemize}[leftmargin=0.15in, label={}]
}
\newcommand{\resumeHeadingListEnd}{\end{itemize}}
%-------------------------------------------

%-----------High School-------------------------
% More reasonable spacing when not including a second title in the quad heading
\newcommand{\resumeSchoolHeading}[5]{
  \item
  \begin{tabular*}{0.96\textwidth}[t]{l@{\extracolsep{\fill}}r}
    \textbf{#1} & #2 \\
    #3 & \textit{\small #4} \\
    #5
  \end{tabular*}
}
%__________________RESUME____________________
% You can rearrange sections in any order you may prefer
\begin{document}

%-----------CONTACT DETAILS------------------
% Make sure all the details are correct, you can add more links in the first row of second column if needed

\begin{tabular*}{\textwidth}{l@{\extracolsep{\fill}}r}
  \textbf{\Huge Struan Robertson \vspace{2pt}} & % row = 1, col = 1
          Flat C, 105B Nethergate, Dundee, Scotland \\ % row = 1, col = 2
  \href{https://struanrobertson.co.uk}{\uline{struanrobertson.co.uk}} $|$ % row = 2, col = 1
  \href{https://linkedin.com/in/struanjrobertson}{\uline{LinkedIn}} $|$ % row = 2, col = 1
  \href{https://github.com/struan-robertson}{\uline{GitHub}} & % row = 2, col = 1
  \href{mailto:contact@struanrobertson.co.uk}{\uline{contact@struanrobertson.co.uk}} $|$ % row = 2, col = 2
  \href{tel:(+44) 478769995}{\uline{(+44) 7478769995}} \\ % row = 2, col = 2
\end{tabular*}
%--------------------------------------------


%-----------SUMMARY--------------------------
% Keep this short, simple and straigth to point

\section{Summary}
\small{
  Applied Computing student with a \textbf{keen interest in AI and Data Science}.
  Strong experience with a \textbf{wide breadth of computing topics} gained through years of personal interest in the subject.
  \textbf{Excellent communication and organisational skills} developed through a range of extra‑curricular activities and customer service roles.
}
%--------------------------------------------


%--------------SKILLS------------------------
% Add or remove resumeSectionTypes according to your needs

\section{Technical Skills}
  \resumeHeadingListStart{}
    \resumeSectionType{Languages}{Rust, Python, C, Scala, Haskell, JavaScript, SQL, Shell, C\#, Nix, LaTeX}
    \resumeSectionType{ML/Data Science}{PyTorch, Numpy, Pandas, Scikit-Learn, GDAL, Jax}
    \resumeSectionType{Big Data Analysis}{Hadoop, Apache Spark, Data Warehousing, Concurrent Systems, Frequent Pattern Mining}
    \resumeSectionType{Back-end}{Linux, OpenBSD, REST API, Azure Functions, Docker}
    \resumeSectionType{Game Development}{Godot, Unity}
  \resumeHeadingListEnd{}
%--------------------------------------------



\section{Education}
  \resumeHeadingListStart{}

    \resumeQuadHeading{University of Dundee}{Dundee, Scotland}
    {BSc in Applied Computing (On track for a first class honors)}{Sep. 2019 -- Jun. 2023}
      \resumeItemListStart{}
        \resumeItem{\textbf{Fourth year modules:} Industial Project, Technology Innovation Management, Big Data Analysis, Research Frontiers, Dissertation (Terrain Model Processing with Machine Learning)}
        \resumeItem{\textbf{Third year modules:} Networks, Information Security, Database Systems, Human Computer Interaction, Agile Software Engineering, Games Programming}
      \resumeItemListEnd{}

    \resumeQuadHeading{UHI West Highland College}{Online -- Fort William, Scotland}
    {Foundation Apprenticeship in Information Technology: Software Development}{Sep. 2017 -- Jun. 2019}

    \resumeSchoolHeading{Mallaig High School}{Mallaig, Highlands, Scotland}
    {}{Aug. 2013 -- Jun. 2019}{
      \begin{minipage}[t]{0.7\textwidth}
        \resumeItemListStart{}
          \resumeItem{\textbf{2019 Higher:} Computer Science \textbf{\textit{A}}, Mathematics \textbf{\textit{A}}, Chemistry \textbf{\textit{B}}}
          \resumeItem{\textbf{2018 Higher:} English \textbf{\textit{B}}}
          \resumeItem{\textbf{2017 Higher:} Physics \textbf{\textit{B}}}
        \resumeHeadingListEnd{}
      \end{minipage}
    }

  \resumeHeadingListEnd{}
%---------------------------------------------


%-----------EXPERIENCE-----------------------
% Distill all your talking points to small bullet points which follow the pattern "challenge-action-result" for maximum efficiency. Try to quantify (use numbers) your points whenever possible, highlight words of importance

\section{Experience}
\resumeHeadingListStart{}
  \resumeQuadHeading{Frontend Developer}{Mar. 2022 -- Jan. 2023}
  {532 Design}{Dundee, Scotland}
    \resumeItemListStart{}
      \resumeItem{Worked part time on a mobile sports game}
      \resumeItem{Implemented using the Unity game engine}
      \resumeItem{Team awarded the Scottish Wild Card EDGE award}
      \resumeItem{Team organised using AGILE methodologies}
      \resumeItem{Coordinated with design team using Figma}
    \resumeItemListEnd{}

  \resumeQuadHeading{Customer Advisor}{Aug. 2021 -- Nov 2021}
  {Hutchinson 3G}{Dundee, Scotland}
    \resumeItemListStart{}
      \resumeItem{Translated technical requirements into sales}
      \resumeItem{Advised customers on technical issues}
      \resumeItem{Coordinated efficietly with a large sales team}
    \resumeItemListEnd{}

  \resumeQuadHeading{Intern}{Mar. 2018 -- Jun. 2019}
  {SiteKit}{Isle of Skye, Scotland}
    \resumeItemListStart{}
      \resumeItem{Interned during Foundation Apprenticeship}
      \resumeItem{Worked on website for defibrillator charity Luck2BHere}
      \resumeItem{Received personal letter from MP Ian Blackford in commendation}
      \resumeItem{Featured in the Oban Times}
    \resumeItemListEnd{}

\resumeHeadingListEnd{}
%---------------------------------------------

%-----------PROJECTS--------------------------
% Use resumeQuadHeading if four elements are feasible (ex: demo video link), else use resumeTrioHeading. Keep the bullet points simple and concise and try to cover wide variety of skills you have used to build these projects

\section{Projects}
  \resumeHeadingListStart{}
    \resumeTrioHeading{Dissertation}{Python, PyTorch, Torchvision, GDAL, Numpy}{\href{https://github.com/struan-robertson/DEM-Void-Infilling}{\uline{Source Code}}}
      \resumeItemListStart{}
        \resumeItem{Implemented a \textbf{Wasserstein GAN with Contextual Attention} for generative infilling of no-data voids in \textbf{Lunar DEMs}}
        \resumeItem{Coded in \textbf{Python}}
        \resumeItem{Network implemented using \textbf{PyTorch}}
        \resumeItem{Image transformations achieved using \textbf{Torchvision}}
        \resumeItem{DEM raster data loaded with \textbf{GDAL} and manipulated using \textbf{Numpy}}
      \resumeItemListEnd{}

      \resumeTrioHeading{Industrial Project}{Node.js, JavaScript, Azure Functions, GitHub Actions, AJAX, JWT}{\href{https://github.com/struan-robertson/compliance-dash-vanilla}{\uline{Source Code}}}
      \resumeItemListStart{}
        \resumeItem{Group project creating an \textbf{admin page} for managing security rules}
        \resumeItem{Developed \textbf{back-end}, \textbf{user authentication} and \textbf{main page}}
        \resumeItem{Back-end implemented using \textbf{serverless Azure Functions}}
        \resumeItem{Azure Functions written in \textbf{JavaScript} and executed using \textbf{Node.js}}
        \resumeItem{Continuous integration using \textbf{GitHub Actions}}
        \resumeItem{User authentication handled using \textbf{rotating refresh tokens} with \textbf{short-lived access tokens}}
        \resumeItem{Client-server communication through \textbf{AJAX} using the \textbf{Axios} JavaScript library}
        \resumeItem{Admin page \textbf{correctly rendered} and \textbf{performant} on a range of devices}
      \resumeItemListEnd{}

      \resumeTrioHeading{Games Programming}{Godot, A* Pathfinding}{\href{https://github.com/struan-robertson/space-pirates}{\uline{Source Code}}}
      \resumeItemListStart{}
        \resumeItem{Game implemented using \textbf{Godot} game engine}
        \resumeItem{Game written in \textbf{GDScript}}
        \resumeItem{\textbf{User-modifiable} game world}
        \resumeItem{Uses \textbf{A* search algorithm} to dynamically pathfind around map changes}
      \resumeItemListEnd{}

      \resumeTrioHeading{Scala Concurrent Wordcount}{Scala, Concurrent Systems, Map Reduce, Akka Actors}{Need to add source}
      \resumeItemListStart{}
        \resumeItem{}
        \resumeItem{}
        \resumeItem{}
        \resumeItem{}
      \resumeItemListEnd{}

  \resumeHeadingListEnd{}
%--------------------------------------------

%--------------REFEREES------------------------
% Add or remove resumeSectionTypes according to your needs

\section{Referees}
  \resumeHeadingListStart{}
    \resumeRefereeHeading{Dr Iain Martin}{\href{mailto:i.martin@dundee.ac.uk}{\uline{i.martin@dundee.ac.uk}}}
    {Dissertation Advisor}{School of Computing, University of Dundee}
    \resumeRefereeHeading{Mr Brian McNicol}{\href{mailto:brian@532design.com}{\uline{brian@532design.com}}}
    {CEO}{532 Design}
  \resumeHeadingListEnd{}
%--------------------------------------------

\end{document}
